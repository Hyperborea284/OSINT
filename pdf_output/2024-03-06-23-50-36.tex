\documentclass[
   article,       
   12pt,          
   oneside,       
   a4paper,       
   english,       
   brazil,        
   sumario=tradicional
   ]{abntex2}

\usepackage{lmodern}       
\usepackage[T1]{fontenc}   
\usepackage[utf8]{inputenc}
\usepackage{indentfirst}   
\usepackage{nomencl}       
\usepackage{color}         
\usepackage{graphicx}      
\usepackage{microtype}     
\usepackage{background}
\usepackage{datetime}
\usepackage{lipsum} 
\usepackage[brazilian,hyperpageref]{backref}
\usepackage[alf]{abntex2cite}

\newdateformat{mydate}{\THEDAY\space de \monthname[\THEMONTH], \THEYEAR}

\backgroundsetup{
   scale=1,
   angle=0,
   opacity=1,
   color=black,
   contents={\begin{tikzpicture}[remember picture, overlay]
      \node at ([xshift=-2cm,yshift=-2cm] current page.north east)
            {\includegraphics[width = 3cm]{logo_02.png}}
       node at ([xshift=2cm,yshift=-2cm] current page.north west)
            {\includegraphics[width = 3cm]{conf.png}};
     \end{tikzpicture}}
}

\renewcommand{\backrefpagesname}{Citado na(s) página(s):~}
\renewcommand{\backref}{}
\renewcommand*{\backrefalt}[4]{
   \ifcase #1
      Nenhuma citação no texto.
   \or
      Citado na página #2.
   \else
      Citado #1 vezes nas páginas #2.
   \fi}

\titulo{Título não encontrado}
\tituloestrangeiro{ }
\autor{{Ephor - Linguística Computacional }}
\local{{Maringá - Brasil \url{https://www.ephor.com.br/}}}
\data{{\today\space \currenttime}}

\definecolor{blue}{RGB}{41,5,195}
\makeatletter
\hypersetup{
      pdftitle={\@title}, 
      pdfauthor={\@author},
      pdfsubject={Correntes da Antropologia},
       pdfcreator={LaTeX with abnTeX2},
      pdfkeywords={abnt}{latex}{abntex}{abntex2}{atigo científico}, 
      colorlinks=true,   
      linkcolor=blue,    
      citecolor=blue,    
      filecolor=magenta, 
      urlcolor=blue,
      bookmarksdepth=4
}
\makeatother
\makeindex
\setlrmarginsandblock{3cm}{3cm}{*}
\setulmarginsandblock{3cm}{3cm}{*}
\checkandfixthelayout
\setlength{\parindent}{1.3cm}
\setlength{\parskip}{0.2cm}
\SingleSpacing

\begin{document}

\selectlanguage{brazil}
\frenchspacing 
\maketitle

\textual
\section{Aviso Importante}
\textbf{Este documento foi gerado usando processamento de linguística computacional auxiliado por inteligência artificial.} Para tanto foram analisadas as seguintes fontes:  \cite{A_CAUSA_E_AS_POLITICAS_DE_DIREITOS_HUMANOS_NO}, \cite{Ciclo_de_Politicas_Publicas_por_que_e_importa}, \cite{Conheca_o_ciclo_das_politicas_publicas__Polit}, \cite{Educacao_Inclusiva_Conheca_o_historico_da_leg}, \cite{Em_Buenos_Aires_Silvio_Almeida_defende_a_inst}, \cite{Entendendo_a_Tipologia_de_Politicas_Publicas_}, \cite{Escola_Nacional_de_Administracao_Publica__Wik}, \cite{Especialista_em_politicas_publicas_e_gestao_g}, \cite{FEDERALISMO_E_POLITICAS_PUBLICAS_NO_BRASIL_Ho}, \cite{Institucionalizacao_das_politicas_em_Direitos}, \cite{Ministerio_do_Planejamento_e_Orcamento__Wikip}, \cite{Ministro_defende_que_direitos_humanos_precisa}, \cite{Politica_conceito_politicas_publicas_e_partid}, \cite{Politica_publica__o_que_e_tipos_de_politicas_}, \cite{Politica_publica__Wikipedia_a_enciclopedia_li}, \cite{Politicas_publicas__Wikipedia_la_enciclopedia}, \cite{Politicas_Publicas_entenda_o_que_sao_para_que}, \cite{Politicas_Publicas_o_que_sao_e_para_que_serve}, \cite{Politicas_publicas_o_que_sao_e_para_que_serve}, \cite{Politicas_publicas_o_que_sao_quem_faz_e_tipos}, \cite{Politicas_publicas_o_que_sao_tipos_e_exemplos}, \cite{Revista_USP_119__Dossie_1_Democracia_e_politi}, \cite{TCU_Ciclo_das_politicas_publicas__Tudo_o_que_}.
\textbf{Portanto este conteúdo requer revisão humana, pois pode conter erros.} Decisões jurídicas, de saúde, financeiras ou similares não devem ser tomadas com base somente neste documento. A Ephor - Linguística Computacional não se responsabiliza por decisões ou outros danos oriundos da tomada de decisão sem a consulta dos devidos especialistas.
A consulta da originalidade deste conteúdo para fins de verificação de plágio pode ser feita em \href{http://www.ephor.com.br}{ephor.com.br}.

\postextual
\bibliography{con_ger_pol_pub_04}
\end{document}