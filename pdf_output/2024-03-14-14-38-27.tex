\documentclass[
   article,       
   12pt,          
   oneside,       
   a4paper,       
   english,       
   brazil,        
   sumario=tradicional
   ]{abntex2}

\usepackage{lmodern}       
\usepackage[T1]{fontenc}   
\usepackage[utf8]{inputenc}
\usepackage{indentfirst}   
\usepackage{nomencl}       
\usepackage{color}         
\usepackage{graphicx}      
\usepackage{microtype}     
\usepackage{background}
\usepackage{datetime}
\usepackage{lipsum} 
\usepackage[brazilian,hyperpageref]{backref}
\usepackage[alf]{abntex2cite}

\newdateformat{mydate}{\THEDAY\space de \monthname[\THEMONTH], \THEYEAR}

\backgroundsetup{
   scale=1,
   angle=0,
   opacity=1,
   color=black,
   contents={\begin{tikzpicture}[remember picture, overlay]
      \node at ([xshift=-2cm,yshift=-2cm] current page.north east)
            {\includegraphics[width = 3cm]{logo_02.png}}
       node at ([xshift=2cm,yshift=-2cm] current page.north west)
            {\includegraphics[width = 3cm]{conf.png}};
     \end{tikzpicture}}
}

\renewcommand{\backrefpagesname}{Citado na(s) página(s):~}
\renewcommand{\backref}{}
\renewcommand*{\backrefalt}[4]{
   \ifcase #1
      Nenhuma citação no texto.
   \or
      Citado na página #2.
   \else
      Citado #1 vezes nas páginas #2.
   \fi}

\titulo{# Ministério do Planejamento e Orçamento}
\tituloestrangeiro{ }
\autor{{Ephor - Linguística Computacional }}
\local{{Maringá - Brasil \url{https://www.ephor.com.br/}}}
\data{{\today\space \currenttime}}

\definecolor{blue}{RGB}{41,5,195}
\makeatletter
\hypersetup{
      pdftitle={\@title}, 
      pdfauthor={\@author},
      pdfsubject={Correntes da Antropologia},
       pdfcreator={LaTeX with abnTeX2},
      pdfkeywords={abnt}{latex}{abntex}{abntex2}{atigo científico}, 
      colorlinks=true,   
      linkcolor=blue,    
      citecolor=blue,    
      filecolor=magenta, 
      urlcolor=blue,
      bookmarksdepth=4
}
\makeatother
\makeindex
\setlrmarginsandblock{3cm}{3cm}{*}
\setulmarginsandblock{3cm}{3cm}{*}
\checkandfixthelayout
\setlength{\parindent}{1.3cm}
\setlength{\parskip}{0.2cm}
\SingleSpacing

\begin{document}

\selectlanguage{brazil}
\frenchspacing 
\maketitle

\textual
\section{Aviso Importante}
\textbf{Este documento foi gerado usando processamento de linguística computacional auxiliado por inteligência artificial.} Para tanto foram analisadas as seguintes fontes:  \cite{A_CAUSA_E_AS_POLITICAS_DE_DIREITOS_HUMANOS_NO}, \cite{Ciclo_de_Politicas_Publicas_por_que_e_importa}, \cite{Conheca_o_ciclo_das_politicas_publicas__Polit}, \cite{Educacao_Inclusiva_Conheca_o_historico_da_leg}, \cite{Em_Buenos_Aires_Silvio_Almeida_defende_a_inst}, \cite{Entendendo_a_Tipologia_de_Politicas_Publicas_}, \cite{Escola_Nacional_de_Administracao_Publica__Wik}, \cite{Especialista_em_politicas_publicas_e_gestao_g}, \cite{FEDERALISMO_E_POLITICAS_PUBLICAS_NO_BRASIL_Ho}, \cite{Institucionalizacao_das_politicas_em_Direitos}, \cite{Ministerio_do_Planejamento_e_Orcamento__Wikip}, \cite{Ministro_defende_que_direitos_humanos_precisa}, \cite{Politica_conceito_politicas_publicas_e_partid}, \cite{Politica_publica__o_que_e_tipos_de_politicas_}, \cite{Politica_publica__Wikipedia_a_enciclopedia_li}, \cite{Politicas_publicas__Wikipedia_la_enciclopedia}, \cite{Politicas_Publicas_entenda_o_que_sao_para_que}, \cite{Politicas_Publicas_o_que_sao_e_para_que_serve}, \cite{Politicas_publicas_o_que_sao_e_para_que_serve}, \cite{Politicas_publicas_o_que_sao_quem_faz_e_tipos}, \cite{Politicas_publicas_o_que_sao_tipos_e_exemplos}, \cite{Revista_USP_119__Dossie_1_Democracia_e_politi}, \cite{TCU_Ciclo_das_politicas_publicas__Tudo_o_que_}.
\textbf{Portanto este conteúdo requer revisão humana, pois pode conter erros.} Decisões jurídicas, de saúde, financeiras ou similares não devem ser tomadas com base somente neste documento. A Ephor - Linguística Computacional não se responsabiliza por decisões ou outros danos oriundos da tomada de decisão sem a consulta dos devidos especialistas.
A consulta da originalidade deste conteúdo para fins de verificação de plágio pode ser feita em \href{http://www.ephor.com.br}{ephor.com.br}.
# Ministério do Planejamento e Orçamento

## Introdução

O Ministério do Planejamento e Orçamento (MPO) desempenha um papel fundamental no Poder Executivo do Brasil, atuando em diversas áreas que contribuem para o planejamento e a gestão governamental.

## A Causa e as Políticas de Direitos Humanos

O Ministério do Planejamento e Orçamento (MPO) tem um papel relevante na avaliação dos impactos socioeconômicos das políticas e programas do Governo Federal, além de ser responsável pela elaboração de estudos especiais para a reformulação de políticas. [1]

## Ciclo de Políticas Públicas: Por que e Importa?

O MPO desempenha um papel crucial na formulação de diretrizes, coordenação das negociações, acompanhamento e avaliação dos financiamentos externos de projetos públicos com organismos multilaterais e agências governamentais. [1]

## Conheça o Ciclo das Políticas Públicas: Política de

O Ministério do Planejamento e Orçamento participa ativamente na coordenação e gestão dos sistemas de planejamento e orçamento federal, entre outras funções importantes para a gestão governamental. [1]

A avaliação dos impactos socioeconômicos das políticas e programas do Governo federal e elaboração de estudos especiais para a reformulação de políticas; Realização de estudos e pesquisas para acompanhamento da conjuntura socioeconômica e gestão dos sistemas cartográficos e estatísticos nacionais; Formulação de diretrizes, coordenação das negociações, acompanhamento e avaliação dos financiamentos externos de projetos públicos com organismos multilaterais e agências governamentais; Coordenação e gestão dos sistemas de planejamento e orçamento federal, de pessoal civil, de administração de recursos da informação e informática e de serviços gerais, bem como.

### Referências

[1] Ministério do Planejamento e Orçamento. Disponível em: A_CAUSA_E_AS_POLITICAS_DE_DIREITOS_HUMANOS_NO Ciclo_de_Politicas_Publicas_por_que_e_importa Conheca_o_ciclo_das_politicas_publicas__Polit Educacao_Inclusiva_Conheca_o_historico_da_leg Em_Buenos_Aires_Silvio_Almeida_defende_a_inst Entendendo_a_Tipologia_de_Politicas_Publicas_ Escola_Nacional_de_Administracao_Publica__Wik Especialista_em_politicas_publicas_e_gestao_g FEDERALISMO_E_POLITICAS_PUBLICAS_NO_BRASIL_Ho Institucionalizacao_das_politicas_em_Direitos Ministerio_do_Planejamento_e_Orcamento__Wikip Ministro_defende_que_direitos_humanos_precisa Politica_conceito_politicas_publicas_e_partid Politica_publica__o_que_e_tipos_de_politicas_ Politica_publica__Wikipedia_a_enciclopedia_li Politicas_publicas__Wikipedia_la_enciclopedia Politicas_Publicas_entenda_o_que_sao_para_que Politicas_Publicas_o_que_sao_e_para_que_serve Politicas_publicas_o_que_sao_e_para_que_serve Politicas_publicas_o_que_sao_quem_faz_e_tipos Politicas_publicas_o_que_sao_tipos_e_exemplos Revista_USP_119__Dossie_1_Democracia_e_politi TCU_Ciclo_das_politicas_publicas__Tudo_o_que_

[2] Ministério do Planejamento e Orçamento (MPO). Disponível em: [link]

[3] Informações sobre a atual ministra do Planejamento e Orçamento. Disponível em: [link]

[4] Anúncio da criação do Ministério da Economia. Disponível em: [link]
\section{Entidades mencionadas}
\subsection{Pessoas jurídicas}
\begin{itemize}
    \item Ministério do Planejamento e Orçamento (MPO) [2]
    \item Governo Federal
\end{itemize}

\subsection{Pessoas físicas}
\begin{itemize}
    \item Simone Tebet - Ministra do Planejamento e Orçamento
    \item Jair Bolsonaro - Presidente do Brasil
    \item Luiz Inácio Lula da Silva - Presidente do Brasil
    \item João Goulart - Ex-presidente do Brasil
    \item Celso Furtado - Ex-ministro do Planejamento e Orçamento
    \item Roberto Campos - Ex-ministro do Planejamento e Orçamento
\end{itemize}

\subsection{Órgãos e organismos}
\begin{itemize}
    \item Ministério da Economia
    \item Ministério da Fazenda
    \item Ministério do Planejamento, Desenvolvimento e Gestão
    \item Ministério da Indústria, Comércio Exterior e Serviços
\end{itemize}

\subsection{Outros sujeitos}
\begin{itemize}
    \item Projetos do governo
    \item Estados
    \item Organismos multilaterais e agências governamentais
\end{itemize}
Para elaborarmos um texto que compila as informações fornecidas, devemos dividir o assunto em seções lógicas e conectar as informações temporais apresentadas. As seções e subseções devem se basear nas informações fornecidas sobre o Ministério do Planejamento e Orçamento, incluindo sua criação, extinção e recriação, bem como a função do ministério.

\section{Criação do Ministério do Planejamento e Orçamento}

O Ministério do Planejamento e Orçamento (MPO) foi criado em 1962 durante o governo João Goulart, com o objetivo de planejar a administração governamental, analisar a viabilidade de projetos, controlar orçamentos e liberar fundos para estados e projetos do governo. Durante o período de João Goulart, o ministro deste período foi Celso Furtado e o programa lançado pelo ministério foi o Plano Trienal.

\section{Extinção e Reabertura}

Pouco mais de um ano após sua criação, o Ministério do Planejamento foi fechado em 31 de março de 1964, durante o governo Castelo Branco. No entanto, meses depois, ainda em 1964, foi reaberto, tendo como seu primeiro ministro desse período Roberto Campos. O primeiro programa desta nova fase foi o Programa de Ação Econômica do Governo (PAEG).

\section{Extinção e Recriação Recentes}

Em 30 de outubro de 2018, foi anunciada a extinção do Ministério do Planejamento e Orçamento pelo então presidente eleito Jair Bolsonaro. Foi criado o Ministério da Economia, resultante da fusão dos ministérios da Fazenda, do Planejamento, Desenvolvimento e Gestão e da Indústria, Comércio Exterior e Serviços. No entanto, em 1 de janeiro de 2023, durante o governo de Luiz Inácio Lula da Silva, o Ministério do Planejamento e Orçamento foi recriado, e a atual ministra é Simone Tebet.

\section{Funções do Ministério do Planejamento e Orçamento}

As funções do Ministério do Planejamento e Orçamento incluem a avaliação dos impactos socioeconômicos das políticas e programas do governo federal, a elaboração de estudos especiais para a reformulação de políticas, realização de estudos e pesquisas para acompanhamento da conjuntura socioeconômica, gestão dos sistemas cartográficos e estatísticos nacionais, formulação de diretrizes e coordenação de financiamentos externos de projetos públicos, e a coordenação e gestão de sistemas de planejamento e orçamento federal, pessoal civil, administração de recursos da informação e informática, e de serviços gerais.
**Ministério do Planejamento e Orçamento:** O Ministério do Planejamento e Orçamento (MPO) [2] exerce a função de planejar a administração governamental, custos, viabilidade de projetos, controle de orçamentos e liberação de fundos para estados e projetos do governo. Criado em 1962 durante o governo João Goulart, foi fechado pouco mais de um ano depois, reaberto já no governo Castelo Branco [3]. Durante o governo Bolsonaro, foi extinto e recriado em 2023 pelo presidente Luiz Inácio Lula da Silva. O atual ministra do MPO é Simone Tebet. Em 2018, o então presidente eleito Jair Bolsonaro anunciou a criação do Ministério da Economia, resultando na fusão de diferentes ministérios [4].

**Principais atribuições:**
- Avaliação dos impactos socioeconômicos das políticas e programas do Governo federal.
- Elaboração de estudos especiais para a reformulação de políticas.
- Realização de estudos e pesquisas para acompanhamento da conjuntura socioeconômica e gestão dos sistemas cartográficos e estatísticos nacionais.
- Formulação de diretrizes, coordenação das negociações e avaliação dos financiamentos externos de projetos públicos com organismos multilaterais e agências governamentais.
- Coordenação e gestão dos sistemas de planejamento e orçamento federal, de pessoal civil, de administração de recursos da informação e informática e de serviços gerais.
Os textos tratam de temáticas relacionadas a politicas públicas e gestão governamental. O Ministério do Planejamento e Orçamento é apresentado como um órgão responsável pelo planejamento, administração governamental, análise de projetos, controle de orçamentos e liberação de fundos para estados e projetos do governo. A extinção e recriação do Ministério em diferentes governos é mencionada. Além disso, são descritas as atribuições do Ministério, como a avaliação de impactos socioeconômicos das políticas do governo, estudos de conjuntura socioeconômica, gestão de sistemas de planejamento e orçamento federal, entre outras funções. Em suma, os textos abordam a importância da gestão governamental na formulação e implementação de políticas públicas, além da relevância do planejamento e controle orçamentário para o funcionamento eficaz do governo.
Desculpe, mas não compreendi claramente sua solicitação. Você poderia fornecer mais detalhes ou exemplos para que eu possa entender melhor e lhe ajudar da melhor maneira possível?
\postextual
\bibliography{con_ger_pol_pub_04}
\end{document}