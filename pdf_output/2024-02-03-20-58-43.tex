\documentclass[
   article,       
   12pt,          
   oneside,       
   a4paper,       
   english,       
   brazil,        
   sumario=tradicional
   ]{abntex2}

\usepackage{lmodern}       
\usepackage[T1]{fontenc}   
\usepackage[utf8]{inputenc}
\usepackage{indentfirst}   
\usepackage{nomencl}       
\usepackage{color}         
\usepackage{graphicx}      
\usepackage{microtype}     
\usepackage{background}
\usepackage{datetime}
\usepackage{lipsum} 
\usepackage[brazilian,hyperpageref]{backref}
\usepackage[alf]{abntex2cite}

\newdateformat{mydate}{\THEDAY\space de \monthname[\THEMONTH], \THEYEAR}

\backgroundsetup{
   scale=1,
   angle=0,
   opacity=1,
   color=black,
   contents={\begin{tikzpicture}[remember picture, overlay]
      \node at ([xshift=-2cm,yshift=-2cm] current page.north east)
            {\includegraphics[width = 3cm]{logo_02.png}}
       node at ([xshift=2cm,yshift=-2cm] current page.north west)
            {\includegraphics[width = 3cm]{conf.png}};
     \end{tikzpicture}}
}

\renewcommand{\backrefpagesname}{Citado na(s) página(s):~}
\renewcommand{\backref}{}
\renewcommand*{\backrefalt}[4]{
   \ifcase #1
      Nenhuma citação no texto.
   \or
      Citado na página #2.
   \else
      Citado #1 vezes nas páginas #2.
   \fi}

\titulo{Políticas Públicas no Brasil}
\tituloestrangeiro{ }
\autor{{Ephor - Linguística Computacional }}
\local{{Maringá - Brasil \url{https://www.ephor.com.br/}}}
\data{{\today\space \currenttime}}

\definecolor{blue}{RGB}{41,5,195}
\makeatletter
\hypersetup{
      pdftitle={\@title}, 
      pdfauthor={\@author},
      pdfsubject={Correntes da Antropologia},
       pdfcreator={LaTeX with abnTeX2},
      pdfkeywords={abnt}{latex}{abntex}{abntex2}{atigo científico}, 
      colorlinks=true,   
      linkcolor=blue,    
      citecolor=blue,    
      filecolor=magenta, 
      urlcolor=blue,
      bookmarksdepth=4
}
\makeatother
\makeindex
\setlrmarginsandblock{3cm}{3cm}{*}
\setulmarginsandblock{3cm}{3cm}{*}
\checkandfixthelayout
\setlength{\parindent}{1.3cm}
\setlength{\parskip}{0.2cm}
\SingleSpacing

\begin{document}

\selectlanguage{brazil}
\frenchspacing 
\maketitle

\textual
\section{Aviso Importante}
\textbf{Este documento foi gerado usando processamento de linguística computacional auxiliado por inteligência artificial.} Para tanto foram analisadas as seguintes fontes:  \cite{A_CAUSA_E_AS_POLITICAS_DE_DIREITOS_HUMANOS_NO}, \cite{Ciclo_de_Politicas_Publicas_por_que_e_importa}, \cite{Conheca_o_ciclo_das_politicas_publicas__Polit}, \cite{Educacao_Inclusiva_Conheca_o_historico_da_leg}, \cite{Entendendo_a_Tipologia_de_Politicas_Publicas_}, \cite{Escola_Nacional_de_Administracao_Publica__Wik}, \cite{Especialista_em_politicas_publicas_e_gestao_g}, \cite{FEDERALISMO_E_POLITICAS_PUBLICAS_NO_BRASIL_Ho}, \cite{Ministerio_do_Planejamento_e_Orcamento__Wikip}, \cite{Ministro_defende_que_direitos_humanos_precisa}, \cite{Politica_conceito_politicas_publicas_e_partid}, \cite{Politica_publica__o_que_e_tipos_de_politicas_}, \cite{Politica_publica__Wikipedia_a_enciclopedia_li}, \cite{Politicas_publicas__Wikipedia_la_enciclopedia}, \cite{Politicas_Publicas_entenda_o_que_sao_para_que}, \cite{Politicas_Publicas_o_que_sao_e_para_que_serve}, \cite{Politicas_publicas_o_que_sao_e_para_que_serve}, \cite{Politicas_publicas_o_que_sao_quem_faz_e_tipos}, \cite{Politicas_publicas_o_que_sao_tipos_e_exemplos}, \cite{Revista_USP_119__Dossie_1_Democracia_e_politi}, \cite{TCU_Ciclo_das_politicas_publicas__Tudo_o_que_}.
\textbf{Portanto este conteúdo requer revisão humana, pois pode conter erros.} Decisões jurídicas, de saúde, financeiras ou similares não devem ser tomadas com base somente neste documento. A Ephor - Linguística Computacional não se responsabiliza por decisões ou outros danos oriundos da tomada de decisão sem a consulta dos devidos especialistas.
A consulta da originalidade deste conteúdo para fins de verificação de plágio pode ser feita em \href{http://www.ephor.com.br}{ephor.com.br}.
\section{Introdução}
Este guia é dedicado a explorar o complexo mundo das políticas públicas no Brasil, um tema de suma importância que afeta a vida de todos os cidadãos brasileiros. As políticas públicas englobam uma série de programas, ações e decisões governamentais que buscam assegurar direitos e promover o bem-estar da sociedade em diversas áreas como saúde, educação, meio ambiente, entre outras. A compreensão deste tema é essencial para entender como o planejamento e a implementação das políticas públicas ocorrem no país, além de compreender a diferença entre políticas de Estado e políticas de governo.\textbackslash{}section\{O que são Políticas Públicas?\}\\Políticas públicas podem ser definidas como conjuntos de programas, ações e decisões tomadas pelos governos, com a participação de entidades públicas e privadas, visando garantir direitos de cidadania a diferentes grupos da sociedade. Tais políticas são fundamentais para o desenvolvimento social e econômico do país e estão diretamente relacionadas à qualidade de vida da população. Elas podem ser analisadas sob dois aspectos principais: como um processo de decisão política, envolvendo conflitos de interesses, e como um conjunto de atividades administrativas, onde se planeja e se executa projetos e programas governamentais.
\section{Argumentos}
Políticas Públicas no Brasil\textbackslash{}section\{Introdução\}Este guia é dedicado a explorar o complexo mundo das políticas públicas no Brasil, um tema de suma importância que afeta a vida de todos os cidadãos brasileiros. As políticas públicas englobam uma série de programas, ações e decisões governamentais que buscam assegurar direitos e promover o bem-estar da sociedade em diversas áreas como saúde, educação, meio ambiente, entre outras. A compreensão deste tema é essencial para entender como o planejamento e a implementação das políticas públicas ocorrem no país, além de compreender a diferença entre políticas de Estado e políticas de governo.\textbackslash{}section\{O que são Políticas Públicas?\}Políticas públicas podem ser definidas como conjuntos de programas, ações e decisões tomadas pelos governos, com a participação de entidades públicas e privadas, visando garantir direitos de cidadania a diferentes grupos da sociedade. Tais políticas são fundamentais para o desenvolvimento social e econômico do país e estão diretamente relacionadas à qualidade de vida da população. Elas podem ser analisadas sob dois aspectos principais: como um processo de decisão política, envolvendo conflitos de interesses, e como um conjunto de atividades administrativas, onde se planeja e se executa projetos e programas governamentais.\textbackslash{}section\{Diferença entre Política de Estado e Política de Governo\}\textbackslash{}subsection\{Política de Estado\}Refere-se às políticas que, independente do governo do momento, devem ser continuadas por estarem amparadas pela Constituição. Elas transcendem os mandatos e possuem uma visão de longo prazo para o desenvolvimento do país.\textbackslash{}subsection\{Política de Governo\}São políticas específicas de um determinado governo, podendo ser modificadas ou descontinuadas com a mudança do poder. Estas políticas refletem os projetos e planos da gestão vigente e podem variar de acordo com as ideologias e prioridades do governo.\textbackslash{}section\{Entidades\}\textbackslash{}subsection\{Indivíduos\}\textbackslash{}begin\{itemize\}    \textbackslash{}item Aécio Neves: Líder oposicionista que, em 2014, propôs tornar o programa Bolsa Família uma política de Estado.    \textbackslash{}item Leonardo Secchi: Especialista em políticas públicas que colaborou com um vídeo sobre o tema.\textbackslash{}end\{itemize\}\textbackslash{}subsection\{Entidades\}\textbackslash{}begin\{itemize\}    \textbackslash{}item Governo Federal, Estaduais e Municipais: Responsáveis pela formulação e implementação das políticas públicas no Brasil.    \textbackslash{}item Constituição Federal: Documento que ampara as políticas de Estado e garante os direitos universais da cidadania.\textbackslash{}end\{itemize\}\textbackslash{}section\{Linha do Tempo\}\textbackslash{}subsection\{Eventos\}\textbackslash{}begin\{itemize\}    \textbackslash{}item Criação do Programa Bolsa Família: Exemplo de política pública com potencial para se tornar política de Estado.    \textbackslash{}item Proposta de Aécio Neves em 2014: Iniciativa para incluir o Bolsa Família na Lei Orgânica da Assistência Social, visando torná-lo uma política de Estado.\textbackslash{}end\{itemize\}\textbackslash{}section\{Contradições\}\textbackslash{}subsection\{Argumentos Positivos\}\textbackslash{}begin\{itemize\}    \textbackslash{}item As políticas públicas são essenciais para a garantia de direitos e para o desenvolvimento social e econômico do país.    \textbackslash{}item Políticas de Estado promovem uma continuidade nas ações governamentais que são fundamentais para um planejamento de longo prazo.\textbackslash{}end\{itemize\}\textbackslash{}subsection\{Argumentos Negativos\}\textbackslash{}begin\{itemize\}    \textbackslash{}item A alternância de poder pode resultar na descontinuação de políticas públicas importantes, afetando o progresso dessas iniciativas.    \textbackslash{}item Há uma percepção de centralização e falta de transparência nas decisões governamentais que afetam a eficácia das políticas públicas.\textbackslash{}end\{itemize\}\textbackslash{}section\{Conclusão\}As políticas públicas representam um pilar fundamental para a democracia e para a promoção do bem-estar social no Brasil. Elas encapsulam a essência do trabalho governamental e sua interação com a sociedade, buscando atender às necessidades dos cidadãos em diversas áreas. Entender a diferença entre políticas de Estado e de governo, bem como conhecer os processos por trás do planejamento e implementação dessas políticas, é crucial para qualquer cidadão que deseje compreender e participar ativamente da vida pública do país.\textbackslash{}section\{Questão 1\}Qual é a principal diferença entre política de Estado e política de governo?\textbackslash{}itemize    \textbackslash{}item Política de governo possui visão de longo prazo e não muda com governos.    \textbackslash{}item Política de Estado é formulada exclusivamente pelo poder executivo.    \textbackslash{}item Política de Estado transcende os mandatos governamentais e é amparada pela Constituição.    \textbackslash{}item Política de governo está relacionada apenas a aspectos econômicos do país.\textbackslash{}subsection\{Resposta:\} Política de Estado transcende os mandatos governamentais e é amparada pela Constituição.\textbackslash{}section\{Questão 2\}O que são políticas públicas?\textbackslash{}itemize    \textbackslash{}item Decisões e ações exclusivas do setor privado para o bem social.    \textbackslash{}item Programas de entretenimento financiados pelo governo.    \textbackslash{}item Conjuntos de programas, ações e decisões governamentais para assegurar direitos de cidadania.    \textbackslash{}item Iniciativas internacionais sem participação governamental.\textbackslash{}subsection\{Resposta:\} Conjuntos de programas, ações e decisões governamentais para assegurar direitos de cidadania.\textbackslash{}section\{Questão 3\}Quais são os dois sentidos em que o conceito de políticas públicas pode ser analisado?\textbackslash{}itemize    \textbackslash{}item Como um processo de decisão política e como um fenômeno natural.    \textbackslash{}item Como um conjunto de regras jurídicas e como uma prática social.    \textbackslash{}item Como um processo de decisão política e como um conjunto de atividades administrativas.    \textbackslash{}item Como uma estratégia de marketing governamental e como uma política econômica.\textbackslash{}subsection\{Resposta:\} Como um processo de decisão política e como um conjunto de atividades administrativas.\textbackslash{}section\{Questão 4\}Qual a importância das políticas públicas para a sociedade?\textbackslash{}itemize    \textbackslash{}item Promover o entretenimento e a cultura pop exclusivamente.    \textbackslash{}item Garantir direitos de cidadania e promover o bem-estar social em diversas áreas.    \textbackslash{}item Centralizar o poder nas mãos do governo, minimizando a participação popular.    \textbackslash{}item Reduzir a influência das organizações não governamentais e do terceiro setor.\textbackslash{}subsection\{Resposta:\} Garantir direitos de cidadania e promover o bem-estar social em diversas áreas.\textbackslash{}section\{Questão 5\}O que é necessário para que uma política pública seja eficiente?\textbackslash{}itemize    \textbackslash{}item Apoio exclusivo do setor privado e desregulamentação completa.    \textbackslash{}item Decisões governamentais centralizadas e falta de transparência.    \textbackslash{}item Participação da comunidade e transparência nas decisões governamentais.    \textbackslash{}item Implementação de políticas exclusivamente em períodos eleitorais.\textbackslash{}subsection\{Resposta:\} Participação da comunidade e transparência nas decisões governamentais.\textbackslash{}section\{Questão 6\}Qual papel as políticas públicas desempenham na democracia?\textbackslash{}itemize    \textbackslash{}item Diminuem a importância do voto popular.    \textbackslash{}item Impedem a participação direta da população na gestão pública.    \textbackslash{}item Atuam como mecanismo de promoção do bem-estar social e garantia de direitos.    \textbackslash{}item Servem apenas como ferramenta de propaganda política.\textbackslash{}subsection\{Resposta:\} Atuam como mecanismo de promoção do bem-estar social e garantia de direitos.\textbackslash{}section\{Questão 7\}Como a comunidade pode participar da formulação de políticas públicas?\textbackslash{}itemize    \textbackslash{}item Apenas por meio de votação em eleições gerais.    \textbackslash{}item Exclusivamente através de doações financeiras para campanhas políticas.    \textbackslash{}item Participando de audiências públicas e conselhos comunitários.    \textbackslash{}item A comunidade não tem permissão para participar da formulação de políticas públicas.\textbackslash{}subsection\{Resposta:\} Participando de audiências públicas e conselhos comunitários.\textbackslash{}section\{Questão 8\}O que um programa de transferência de renda exemplifica nas políticas públicas?\textbackslash{}itemize    \textbackslash{}item Uma política de governo temporária e de impacto limitado.    \textbackslash{}item Um mecanismo desenhado para centralizar o poder econômico.    \textbackslash{}item Uma política pública visando assegurar direitos e reduzir desigualdades.    \textbackslash{}item Uma estratégia para aumentar a dependência da população ao governo.\textbackslash{}subsection\{Resposta:\} Uma política pública visando assegurar direitos e reduzir desigualdades.\textbackslash{}section\{Questão 9\}Qual é o papel dos governos federal, estaduais e municipais nas políticas públicas?\textbackslash{}itemize    \textbackslash{}item Executar políticas públicas sem a necessidade de cooperação ou coordenação.    \textbackslash{}item Promover apenas políticas de interesse econômico internacional.    \textbackslash{}item Formular e implementar políticas públicas em diferentes níveis de governo.    \textbackslash{}item Limitar a participação da sociedade civil na formulação de políticas públicas.\textbackslash{}subsection\{Resposta:\} Formular e implementar políticas públicas em diferentes níveis de governo.\textbackslash{}section\{Questão 10\}Por que as políticas públicas precisam de transparência em sua formulação e execução?\textbackslash{}itemize    \textbackslash{}item Para aumentar a burocracia e o tempo de implementação de programas.    \textbackslash{}item Para limitar a participação popular e focar em interesses específicos.    \textbackslash{}item Para assegurar a confiança pública e promover a participação cidadã.    \textbackslash{}item Para diminuir a eficácia das políticas públicas e aumentar custos.\textbackslash{}subsection\{Resposta:\} Para assegurar a confiança pública e promover a participação cidadã.


\section{Diferença entre Política de Estado e Política de Governo}
\subsection{Política de Estado}
Refere-se às políticas que, independente do governo do momento, devem ser continuadas por estarem amparadas pela Constituição. Elas transcendem os mandatos e possuem uma visão de longo prazo para o desenvolvimento do país.
\subsection{Política de Governo}
São políticas específicas de um determinado governo, podendo ser modificadas ou descontinuadas com a mudança do poder. Estas políticas refletem os projetos e planos da gestão vigente e podem variar de acordo com as ideologias e prioridades do governo.
\section{Entidades}
\subsection{Indivíduos}
\begin{itemize}
    \item Aécio Neves: Líder oposicionista que, em 2014, propôs tornar o programa Bolsa Família uma política de Estado.
    \item Leonardo Secchi: Especialista em políticas públicas que colaborou com um vídeo sobre o tema.
\end{itemize}
\subsection{Entidades}
\begin{itemize}
    \item Governo Federal, Estaduais e Municipais: Responsáveis pela formulação e implementação das políticas públicas no Brasil.
    \item Constituição Federal: Documento que ampara as políticas de Estado e garante os direitos universais da cidadania.
\end{itemize}
\section{Linha do Tempo}
\subsection{Eventos}
\begin{itemize}
    \item Criação do Programa Bolsa Família: Exemplo de política pública com potencial para se tornar política de Estado.
    \item Proposta de Aécio Neves em 2014: Iniciativa para incluir o Bolsa Família na Lei Orgânica da Assistência Social, visando torná-lo uma política de Estado.
\end{itemize}
\section{Contradições}
\subsection{Argumentos Positivos}
\begin{itemize}
    \item As políticas públicas são essenciais para a garantia de direitos e para o desenvolvimento social e econômico do país.
    \item Políticas de Estado promovem uma continuidade nas ações governamentais que são fundamentais para um planejamento de longo prazo.
\end{itemize}
\subsection{Argumentos Negativos}
\begin{itemize}
    \item A alternância de poder pode resultar na descontinuação de políticas públicas importantes, afetando o progresso dessas iniciativas.
    \item Há uma percepção de centralização e falta de transparência nas decisões governamentais que afetam a eficácia das políticas públicas.
\end{itemize}
\section{Conclusão}
As políticas públicas representam um pilar fundamental para a democracia e para a promoção do bem-estar social no Brasil. Elas encapsulam a essência do trabalho governamental e sua interação com a sociedade, buscando atender às necessidades dos cidadãos em diversas áreas. Entender a diferença entre políticas de Estado e de governo, bem como conhecer os processos por trás do planejamento e implementação dessas políticas, é crucial para qualquer cidadão que deseje compreender e participar ativamente da vida pública do país.
\section{Questão 1}
Qual é a principal diferença entre política de Estado e política de governo?
\itemize
    \item Política de governo possui visão de longo prazo e não muda com governos.
    \item Política de Estado é formulada exclusivamente pelo poder executivo.
    \item Política de Estado transcende os mandatos governamentais e é amparada pela Constituição.
    \item Política de governo está relacionada apenas a aspectos econômicos do país.
\subsection{Resposta:} Política de Estado transcende os mandatos governamentais e é amparada pela Constituição.
\section{Questão 2}
O que são políticas públicas?
\itemize
    \item Decisões e ações exclusivas do setor privado para o bem social.
    \item Programas de entretenimento financiados pelo governo.
    \item Conjuntos de programas, ações e decisões governamentais para assegurar direitos de cidadania.
    \item Iniciativas internacionais sem participação governamental.
\subsection{Resposta:} Conjuntos de programas, ações e decisões governamentais para assegurar direitos de cidadania.
\section{Questão 3}
Quais são os dois sentidos em que o conceito de políticas públicas pode ser analisado?
\itemize
    \item Como um processo de decisão política e como um fenômeno natural.
    \item Como um conjunto de regras jurídicas e como uma prática social.
    \item Como um processo de decisão política e como um conjunto de atividades administrativas.
    \item Como uma estratégia de marketing governamental e como uma política econômica.
\subsection{Resposta:} Como um processo de decisão política e como um conjunto de atividades administrativas.
\section{Questão 4}
Qual a importância das políticas públicas para a sociedade?
\itemize
    \item Promover o entretenimento e a cultura pop exclusivamente.
    \item Garantir direitos de cidadania e promover o bem-estar social em diversas áreas.
    \item Centralizar o poder nas mãos do governo, minimizando a participação popular.
    \item Reduzir a influência das organizações não governamentais e do terceiro setor.
\subsection{Resposta:} Garantir direitos de cidadania e promover o bem-estar social em diversas áreas.
\section{Questão 5}
O que é necessário para que uma política pública seja eficiente?
\itemize
    \item Apoio exclusivo do setor privado e desregulamentação completa.
    \item Decisões governamentais centralizadas e falta de transparência.
    \item Participação da comunidade e transparência nas decisões governamentais.
    \item Implementação de políticas exclusivamente em períodos eleitorais.
\subsection{Resposta:} Participação da comunidade e transparência nas decisões governamentais.
\section{Questão 6}
Qual papel as políticas públicas desempenham na democracia?
\itemize
    \item Diminuem a importância do voto popular.
    \item Impedem a participação direta da população na gestão pública.
    \item Atuam como mecanismo de promoção do bem-estar social e garantia de direitos.
    \item Servem apenas como ferramenta de propaganda política.
\subsection{Resposta:} Atuam como mecanismo de promoção do bem-estar social e garantia de direitos.
\section{Questão 7}
Como a comunidade pode participar da formulação de políticas públicas?
\itemize
    \item Apenas por meio de votação em eleições gerais.
    \item Exclusivamente através de doações financeiras para campanhas políticas.
    \item Participando de audiências públicas e conselhos comunitários.
    \item A comunidade não tem permissão para participar da formulação de políticas públicas.
\subsection{Resposta:} Participando de audiências públicas e conselhos comunitários.
\section{Questão 8}
O que um programa de transferência de renda exemplifica nas políticas públicas?
\itemize
    \item Uma política de governo temporária e de impacto limitado.
    \item Um mecanismo desenhado para centralizar o poder econômico.
    \item Uma política pública visando assegurar direitos e reduzir desigualdades.
    \item Uma estratégia para aumentar a dependência da população ao governo.
\subsection{Resposta:} Uma política pública visando assegurar direitos e reduzir desigualdades.
\section{Questão 9}
Qual é o papel dos governos federal, estaduais e municipais nas políticas públicas?
\itemize
    \item Executar políticas públicas sem a necessidade de cooperação ou coordenação.
    \item Promover apenas políticas de interesse econômico internacional.
    \item Formular e implementar políticas públicas em diferentes níveis de governo.
    \item Limitar a participação da sociedade civil na formulação de políticas públicas.
\subsection{Resposta:} Formular e implementar políticas públicas em diferentes níveis de governo.
\section{Questão 10}
Por que as políticas públicas precisam de transparência em sua formulação e execução?
\itemize
    \item Para aumentar a burocracia e o tempo de implementação de programas.
    \item Para limitar a participação popular e focar em interesses específicos.
    \item Para assegurar a confiança pública e promover a participação cidadã.
    \item Para diminuir a eficácia das políticas públicas e aumentar custos.
\subsection{Resposta:} Para assegurar a confiança pública e promover a participação cidadã.

\postextual
\bibliography{con_ger_pol_pub}
\end{document}