\documentclass[
   article,       
   12pt,          
   oneside,       
   a4paper,       
   english,       
   brazil,        
   sumario=tradicional
   ]{abntex2}

\usepackage{lmodern}       
\usepackage[T1]{fontenc}   
\usepackage[utf8]{inputenc}
\usepackage{indentfirst}   
\usepackage{nomencl}       
\usepackage{color}         
\usepackage{graphicx}      
\usepackage{microtype}     
\usepackage{background}
\usepackage{datetime}
\usepackage{lipsum} 
\usepackage[brazilian,hyperpageref]{backref}
\usepackage[alf]{abntex2cite}

\newdateformat{mydate}{\THEDAY\space de \monthname[\THEMONTH], \THEYEAR}

\backgroundsetup{
   scale=1,
   angle=0,
   opacity=1,
   color=black,
   contents={\begin{tikzpicture}[remember picture, overlay]
      \node at ([xshift=-2cm,yshift=-2cm] current page.north east)
            {\includegraphics[width = 3cm]{logo_02.png}}
       node at ([xshift=2cm,yshift=-2cm] current page.north west)
            {\includegraphics[width = 3cm]{conf.png}};
     \end{tikzpicture}}
}

\renewcommand{\backrefpagesname}{Citado na(s) página(s):~}
\renewcommand{\backref}{}
\renewcommand*{\backrefalt}[4]{
   \ifcase #1
      Nenhuma citação no texto.
   \or
      Citado na página #2.
   \else
      Citado #1 vezes nas páginas #2.
   \fi}

\titulo{Políticas Públicas no Brasil}
\tituloestrangeiro{ }
\autor{{Ephor - Linguística Computacional }}
\local{{Maringá - Brasil \url{https://www.ephor.com.br/}}}
\data{{\today\space \currenttime}}

\definecolor{blue}{RGB}{41,5,195}
\makeatletter
\hypersetup{
      pdftitle={\@title}, 
      pdfauthor={\@author},
      pdfsubject={Correntes da Antropologia},
       pdfcreator={LaTeX with abnTeX2},
      pdfkeywords={abnt}{latex}{abntex}{abntex2}{atigo científico}, 
      colorlinks=true,   
      linkcolor=blue,    
      citecolor=blue,    
      filecolor=magenta, 
      urlcolor=blue,
      bookmarksdepth=4
}
\makeatother
\makeindex
\setlrmarginsandblock{3cm}{3cm}{*}
\setulmarginsandblock{3cm}{3cm}{*}
\checkandfixthelayout
\setlength{\parindent}{1.3cm}
\setlength{\parskip}{0.2cm}
\SingleSpacing

\begin{document}

\selectlanguage{brazil}
\frenchspacing 
\maketitle

\textual
\section{Aviso Importante}
\textbf{Este documento foi gerado usando processamento de linguística computacional auxiliado por inteligência artificial.} Para tanto foram analisadas as seguintes fontes:  \cite{A_CAUSA_E_AS_POLITICAS_DE_DIREITOS_HUMANOS_NO}, \cite{Ciclo_de_Politicas_Publicas_por_que_e_importa}, \cite{Conheca_o_ciclo_das_politicas_publicas__Polit}, \cite{Educacao_Inclusiva_Conheca_o_historico_da_leg}, \cite{Entendendo_a_Tipologia_de_Politicas_Publicas_}, \cite{Escola_Nacional_de_Administracao_Publica__Wik}, \cite{Especialista_em_politicas_publicas_e_gestao_g}, \cite{FEDERALISMO_E_POLITICAS_PUBLICAS_NO_BRASIL_Ho}, \cite{Ministerio_do_Planejamento_e_Orcamento__Wikip}, \cite{Ministro_defende_que_direitos_humanos_precisa}, \cite{Politica_conceito_politicas_publicas_e_partid}, \cite{Politica_publica__o_que_e_tipos_de_politicas_}, \cite{Politica_publica__Wikipedia_a_enciclopedia_li}, \cite{Politicas_publicas__Wikipedia_la_enciclopedia}, \cite{Politicas_Publicas_entenda_o_que_sao_para_que}, \cite{Politicas_Publicas_o_que_sao_e_para_que_serve}, \cite{Politicas_publicas_o_que_sao_e_para_que_serve}, \cite{Politicas_publicas_o_que_sao_quem_faz_e_tipos}, \cite{Politicas_publicas_o_que_sao_tipos_e_exemplos}, \cite{Revista_USP_119__Dossie_1_Democracia_e_politi}, \cite{TCU_Ciclo_das_politicas_publicas__Tudo_o_que_}.
\textbf{Portanto este conteúdo requer revisão humana, pois pode conter erros.} Decisões jurídicas, de saúde, financeiras ou similares não devem ser tomadas com base somente neste documento. A Ephor - Linguística Computacional não se responsabiliza por decisões ou outros danos oriundos da tomada de decisão sem a consulta dos devidos especialistas.
A consulta da originalidade deste conteúdo para fins de verificação de plágio pode ser feita em \href{http://www.ephor.com.br}{ephor.com.br}.
\section{Introdução}
Este guia inicia uma jornada pelo universo das políticas públicas, um tema de relevante importância para a compreensão das dinâmicas sociais, econômicas e políticas que moldam a sociedade brasileira. As políticas públicas representam um conjunto de programas, ações e decisões governamentais, os quais são desenhados para atender às necessidades e direitos dos cidadãos em diversos campos como saúde, educação, meio ambiente, entre outros. Este material visa elucidar não apenas o conceito e a importância das políticas públicas, mas também, o seu planejamento, implementação, e diferenciação entre políticas de Estado e de governo.\textbackslash{}section\{O que são Políticas Públicas\}\\Políticas públicas constituem um complexo conjunto de programas, ações e decisões que são tomadas pelos governos nas esferas nacional, estadual ou municipal, com a participação direta ou indireta de entidades tanto públicas quanto privadas. O principal objetivo dessas políticas é assegurar direitos e promover o bem-estar social em diversos segmentos da sociedade. Essencialmente, as políticas públicas são mecanismos por meio dos quais o governo busca intervir em determinadas áreas, visando o desenvolvimento social, cultural, econômico e ambiental da nação.
\section{Argumentos}
Políticas Públicas no Brasil\textbackslash{}section\{Introdução\}Este guia inicia uma jornada pelo universo das políticas públicas, um tema de relevante importância para a compreensão das dinâmicas sociais, econômicas e políticas que moldam a sociedade brasileira. As políticas públicas representam um conjunto de programas, ações e decisões governamentais, os quais são desenhados para atender às necessidades e direitos dos cidadãos em diversos campos como saúde, educação, meio ambiente, entre outros. Este material visa elucidar não apenas o conceito e a importância das políticas públicas, mas também, o seu planejamento, implementação, e diferenciação entre políticas de Estado e de governo.\textbackslash{}section\{O que são Políticas Públicas\}Políticas públicas constituem um complexo conjunto de programas, ações e decisões que são tomadas pelos governos nas esferas nacional, estadual ou municipal, com a participação direta ou indireta de entidades tanto públicas quanto privadas. O principal objetivo dessas políticas é assegurar direitos e promover o bem-estar social em diversos segmentos da sociedade. Essencialmente, as políticas públicas são mecanismos por meio dos quais o governo busca intervir em determinadas áreas, visando o desenvolvimento social, cultural, econômico e ambiental da nação.\textbackslash{}subsection\{Definição e Importância\}Políticas públicas são entendidas como o conjunto de ações e decisões tomadas com o intuito de garantir e promover direitos considerados fundamentais para a convivência coletiva. Elas desempenham um papel fundamental na organização da sociedade e na regulação das relações entre os diferentes grupos e interesses presentes numa comunidade, visando sempre o bem comum.\textbackslash{}subsection\{Planejamento e Implementação\}A criação e efetivação das políticas públicas envolvem um processo minucioso de planejamento, que inclui a identificação das necessidades da população, a formulação de objetivos estratégicos, e a execução de ações direcionadas. Este processo é ciclíco e contínuo, buscando a adaptação e melhoria constantes das políticas em resposta à evolução das demandas sociais.\textbackslash{}subsection\{Políticas de Estado versus Políticas de Governo\}Uma distinção relevante no contexto das políticas públicas é entre políticas de Estado e políticas de governo. As primeiras referem-se a políticas que, por serem fundamentais para a garantia dos direitos constitucionais, devem ser mantidas independentemente das alternâncias de poder. Já as políticas de governo estão mais suscetíveis às mudanças de administração, podendo ser alteradas conforme as diretrizes e prioridades do governante vigente.\textbackslash{}section\{Entidades\}\textbackslash{}subsection\{Indivíduos\}Leonardo Secchi - Especialista em políticas públicas, mencionado por sua contribuição ao tema através de um vídeo em parceria.\textbackslash{}subsection\{Entidades\}Governo Federal, Governos Estaduais, Governos Municipais - Entes responsáveis pela criação, planejamento e implementação de políticas públicas no Brasil.PT (Partido dos Trabalhadores) - Partido político mencionado em relação ao programa Bolsa Família.Aécio Neves - Líder oposicionista que propôs a transformação do Bolsa Família em política de Estado.\textbackslash{}section\{Linha do Tempo\}\textbackslash{}subsection\{Eventos\}Criação e expansão do programa Bolsa Família pelo governo do PT - Exemplo de política pública com potencial de se tornar política de Estado.Proposta de Aécio Neves em 2014 para tornar o Bolsa Família uma política de Estado.\textbackslash{}section\{Contradições\}\textbackslash{}subsection\{Positivos\}As políticas públicas representam uma tentativa de garantir direitos e promover o bem-estar social.\textbackslash{}subsection\{Negativos\}O processo de planejamento e implementação de políticas públicas pode enfrentar desafios como burocracia excessiva, falta de transparência e centralização de poder.\textbackslash{}section\{Conclusão\}As políticas públicas são essenciais para a estruturação e desenvolvimento de uma sociedade que visa o bem-estar coletivo e a garantia de direitos fundamentais. Apesar dos desafios inerentes à sua formulação e execução, entender seu funcionamento, importância, e impacto é crucial para todos os cidadãos. Este guia busca fornecer uma visão abrangente sobre o tema, instigando ao aprofundamento e à participação ativa no processo político do país.\textbackslash{}section\{Questão\}\textbackslash{}itemize\textbackslash{}item\{A\} Políticas públicas são decisões exclusivas do governo federal.\textbackslash{}item\{B\} O Bolsa Família é um exemplo de política de Estado, inalterado por mudanças no governo.\textbackslash{}item\{C\} A criação das políticas públicas não envolve a participação de entidades privadas.\textbackslash{}item\{D\} As políticas públicas podem ser classificadas em políticas de Estado e políticas de governo, dependendo da sua permanência e fundamentação legal.\textbackslash{}subsection\{Resposta\}\textbackslash{}item\{D\} As políticas públicas podem ser classificadas em políticas de Estado e políticas de governo, dependendo da sua permanência e fundamentação legal.


\subsection{Definição e Importância}
Políticas públicas são entendidas como o conjunto de ações e decisões tomadas com o intuito de garantir e promover direitos considerados fundamentais para a convivência coletiva. Elas desempenham um papel fundamental na organização da sociedade e na regulação das relações entre os diferentes grupos e interesses presentes numa comunidade, visando sempre o bem comum.

\subsection{Planejamento e Implementação}
A criação e efetivação das políticas públicas envolvem um processo minucioso de planejamento, que inclui a identificação das necessidades da população, a formulação de objetivos estratégicos, e a execução de ações direcionadas. Este processo é ciclíco e contínuo, buscando a adaptação e melhoria constantes das políticas em resposta à evolução das demandas sociais.

\subsection{Políticas de Estado versus Políticas de Governo}
Uma distinção relevante no contexto das políticas públicas é entre políticas de Estado e políticas de governo. As primeiras referem-se a políticas que, por serem fundamentais para a garantia dos direitos constitucionais, devem ser mantidas independentemente das alternâncias de poder. Já as políticas de governo estão mais suscetíveis às mudanças de administração, podendo ser alteradas conforme as diretrizes e prioridades do governante vigente.
\section{Entidades}
\subsection{Indivíduos}
Leonardo Secchi - Especialista em políticas públicas, mencionado por sua contribuição ao tema através de um vídeo em parceria.
\subsection{Entidades}
Governo Federal, Governos Estaduais, Governos Municipais - Entes responsáveis pela criação, planejamento e implementação de políticas públicas no Brasil.
PT (Partido dos Trabalhadores) - Partido político mencionado em relação ao programa Bolsa Família.
Aécio Neves - Líder oposicionista que propôs a transformação do Bolsa Família em política de Estado.
\section{Linha do Tempo}
\subsection{Eventos}
Criação e expansão do programa Bolsa Família pelo governo do PT - Exemplo de política pública com potencial de se tornar política de Estado.
Proposta de Aécio Neves em 2014 para tornar o Bolsa Família uma política de Estado.
\section{Contradições}
\subsection{Positivos}
As políticas públicas representam uma tentativa de garantir direitos e promover o bem-estar social.
\subsection{Negativos}
O processo de planejamento e implementação de políticas públicas pode enfrentar desafios como burocracia excessiva, falta de transparência e centralização de poder.
\section{Conclusão}
As políticas públicas são essenciais para a estruturação e desenvolvimento de uma sociedade que visa o bem-estar coletivo e a garantia de direitos fundamentais. Apesar dos desafios inerentes à sua formulação e execução, entender seu funcionamento, importância, e impacto é crucial para todos os cidadãos. Este guia busca fornecer uma visão abrangente sobre o tema, instigando ao aprofundamento e à participação ativa no processo político do país.
\section{Questão}
\itemize
\item{A} Políticas públicas são decisões exclusivas do governo federal.
\item{B} O Bolsa Família é um exemplo de política de Estado, inalterado por mudanças no governo.
\item{C} A criação das políticas públicas não envolve a participação de entidades privadas.
\item{D} As políticas públicas podem ser classificadas em políticas de Estado e políticas de governo, dependendo da sua permanência e fundamentação legal.
\subsection{Resposta}
\item{D} As políticas públicas podem ser classificadas em políticas de Estado e políticas de governo, dependendo da sua permanência e fundamentação legal.

\postextual
\bibliography{con_ger_pol_pub}
\end{document}